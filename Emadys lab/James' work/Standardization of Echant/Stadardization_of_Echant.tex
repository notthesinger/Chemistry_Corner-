\documentclass{article}
%--------------------Packages 
\usepackage{ragged2e}
\usepackage{listings}
\usepackage{graphicx}
\usepackage{fancyhdr}
\usepackage{float}
\usepackage{tocloft} % dots in table of content
\usepackage{amsmath} % math stuff
\usepackage[latin1]{inputenc}
% Create a title page
\usepackage{titling}
\usepackage[toc,page]{appendix} % appendix stuff
\usepackage{caption}
\captionsetup[table]{justification=centering,labelfont=it}
%--------------------
% Syntax-highlighted listings
\usepackage{minted}
\usepackage{siunitx}
%--------------------Fancy header stuff
\pagestyle{fancy}
\rhead{Hopper Group}
\lhead{Standardization of Etchant}
\newcommand{\settitle}{Standardization of Etchant}
\newcommand{\setauthor}{James Taylor }
%---------------------------------------------------%
% add dots to table of conetent----------
\renewcommand{\cftsecleader}{\cftdotfill{\cftdotsep}}
% Set document data
\setcounter{secnumdepth}{0}
\title{\settitle}
\author{\setauthor}
\setlength{\parskip}{1em} 
\renewcommand{\contentsname}{Table of Contents}
\renewcommand{\thesubsection}{\alph{subsection}}
\setlength{\parindent}{0 cm}
%----------------------------------------------
%--------Commonly Used Blocks of Code 
% add code
% \inputminted[linenos]{language}{filename.py}
% add figure 
\iffalse
\begin{figure}[H]
\centering
\includegraphics[width=0.7\linewidth]{fig file name.here}
\caption{caption here} 
\label{fig:new fig}
\end{figure}
\fi
%--------------------
\begin{document}

	\begin{titlepage}
	\maketitle
	\tableofcontents
	\thispagestyle{empty}
	\end{titlepage}
	
\newpage
%\section{Introduction}
Insert Intro here.

\section{Materials}
\subsection{Chemicals}
\begin{itemize}
  \item NaOH pellets - Caution corrosive! 
  \item Phenolphthalein indicator solution
  \item Potassium hydrogen phthalate (KHP)
  \item Distilled Water
\end{itemize}
\subsection{Equipment}
\begin{itemize}
  \item 50 ml Burette 
  \item 1 L Plastic storage bottle 
  \item 250 ml beaker
  \item Ring stand
  \item Burette clamp
\end{itemize}
\section{Procedure}
\subsection{Titration}
\begin{enumerate}
  \item Make sodium hydroxide solution by measuring out calculated amount of sodium hydroxide pellets and dissolve the pellets in approximately 900 ml of distilled water. Mix thoroughly. 
  \item Measure out correct amount of KHP on analytical balance, using the reference table below and record the mass. Dissolve in about 100 ml of distilled water in a 250 ml beaker. 
  \item Add 2 to 3 drops of phenolphthalein  to KHP solution.
  \item Prepare burrette by washing with water, then with a small amount of the etching solution that you will be standardizing.   
  \item Add sodium hydroxide solution to burrette, making sure to record the the height of the sodium hydroxide in the burette. 
  \item Add sodium hydroxide until a pink/purple color starts to appear, then  add sodium hydroxide more slowly. 
  \item When the KHP solution stays a light pink or purplish color, the end point is reached. Record the final level of the sodium hydroxide solution.
  \item Calculate the molar concentration(M) of the sodium hydroxide solution.
  \item Run three trials and average the final. 
\end{enumerate}
\newpage
\subsection{Material Calculations}
The above titration procedure will be preformed using the following amounts calculated. These amounts can be changed and methods to do so to fit the present situation will be mentioned.
\subsubsection{Sodium Hydroxide}
The following amounts of distilled water and sodium hydroxide will be needed to make 1 M , 3 M and 6 M solutions of sodium hydroxide. 
  \begin{enumerate}
  \item For 0.25 M solution, 900 ml of distilled water and about 9 grams of sodium hydroxide. 
  \item For 1 M solution, 900 ml of distilled water and about 36 grams of sodium hydroxide. 
  
  \item For 2 M solution, 900 ml of distilled water and about 72 grams of sodium hydroxide. 
\end{enumerate}
\subsubsection{KHP}

The following amount of KHP should be weight out for the corresponding molar solution of sodium hydroxide 
 \begin{enumerate}
  \item For 0.25 M solution, weight out about 0.75 - 0.76 grams of KHP. The end point should be around 15 ml of solution 
  \item For 1 M solution, weight out about 3 - 3.1 grams of KHP. The end point should be around 15 ml of solution 
  \item For 2 M solution, weight out about 6 - 6.1 grams of KHP. The end point should be around 15 ml of solution. 
\end{enumerate}
\end{document}